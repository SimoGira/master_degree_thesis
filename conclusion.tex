\chapter*{Conclusions and future works}
\addcontentsline{toc}{chapter}{Conclusions}
In this thesis a teleoperation architecture has been presented to address the issue of providing kinaesthetic feedback in remote percutaneous procedures such biopsy.\\
The inclusion of a virtual guiding mode for a more precise insertion is only one of the possible solution for augmenting the performance and exploiting such a system.\\
For such kind of system the problem of delay is usually not addressed due to the presence of a dedicated connection between the master console  and the slave robot.
The proposed teleoperated architecture has been tested with positive results even in a constant delayed network.\\
A future, extension of this work could be the test on a time-vary delayed network with packet loss and/or packet retransmission.\\
For an evaluation of the setup architecture more closer to th clinical case, the crioablation needle could be mounted and tested: this will require a re-tuning and possibly a re-shaping of the slave controller.\\
The substitution of the master console with a 6-DoF model, as the Phantom Premium 1.5, will enable the full potentiality of the implemented architecture with no extra effort.\\
With this physical change, a future work could be the evaluation of the system performance in a full bilateral configuration.\\
Another future improvement could be to take into account the dynamical model of the two manipulators, expressed in task space. This should reduce the need for scaling the energy between them and could enable a more suitable control strategy at the slave side for the puncturing task such as the impedance control. 
 
%%%%%%%%

\clearpage
\thispagestyle{empty}