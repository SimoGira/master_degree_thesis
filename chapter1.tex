\chapter{Introduction}
While robots originated in large-scale mass manufacturing working very efficiently behind fences (i.e., not interacting with human operators), they are now spreading to more and more application areas. As these robots operate at human scale (e.g., physical Human-Robot-Interaction in factories, warehouses, and at homes) and perform safety-critical tasks (like driving or surgery), the correctness requirements are stringent and include, beside functional accuracy, also non-functional constraints such as real-time, safety, data privacy, scalability, and energy efficiency \cite{robotchallanges}.

To address all these requirements, robotic SW applications have to be configured to be run across heterogeneous Cloud-Server-Edge computing platforms \cite{deepedgerev}. Such a design and verification task is very challenging, as it requires high-expertise and time-consuming (re)configuration steps to compile and test the application code for different HW/SW configurations.

In this work, we present a toolchain that relies on a container-based packaging mechanism whereby SW components are abstracted from the environment in which they actually run and orchestrated across heterogeneous HW architectures. We extended the Docker and Kubernetes (KubeEdge) environments, which are well-established in the context of Cloud-native SW applications, to be used in the context of ROS-based robotic applications on Cloud-Edge platforms.

Recently, some open source and a number of proprietary model-based platforms have been proposed to ease the design of robotic systems (e.g., the EU funded Robmosys project, NVIDIA Isaac, Amazon AWS Robomaker). What is missing in these solutions is a toolchain able to address the complexity, the scalability, and the heterogeneity of the HW/SW domains by considering non-functional constraints. Taking into account in seamless way functional and non-functional constraints is a key feature to design reliable, and safe robotic applications from the specifications to the system deployment.
This thesis give two main contributions:
\begin{enumerate}
	\item Customization, debugging and verification of the ORB-SLAM2 algorithm for efficient execution at the edge.
	\item A design flow based on containers and KubeEdge, that integrates the ORB-SLAM2 application into a more complex application for mobile robots. 
\end{enumerate}

\section{Thesis outline}
This thesis is organised in two main workflows: \textit{Verification} and \textit{Methodology}.
Chapter \ref{chap:background} is an overview over the research contribution in the field of  deep learning, edge computing and the other tools discussed in the next chapter of this thesis.
The first workflow is described in chapter \ref{chap:verification}, where a complete inspection and verification was made on the ORB SLAM algorithm to guarantee a deterministic behaviour in a sequential context.
The second workflow described in chapter \ref{chap:methodology} proposes a solution to the problem of integrating heterogeneous robotic applications.
With the support of edge computing platforms like KubeEdge, some experiments was made to deploy automatically our integrated system from the host (cloud) to the edge.
Finally in chapter \ref{chap:experimental-results} the experimental results are showed and discussed.

%This thesis arises from a collaboration with the European ICE Lab project assigned to the Computer Science department at the University of Verona.


%\clearpage
\thispagestyle{empty}