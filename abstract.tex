\chapter{Abstract}
\markboth{Abstract}{Abstract}
Developing distributed robotics applications on embedded devices, we have to deal with the diversity of the applications and the different platforms where these applications run.
At the state of the art there are some solitutions that allow us to develop robotics applications and deploy them on embedded boards. 
The problem is that none of these solutions allows us to be sufficiently accurate to guarantee the functioning of the entire system, esapecially if we want to increase its complexity.
To solve the problem we must to take account of the necessary reasources to run the applications and the constraints imposed by the limits of the devices.



%words like edge computing and cluster of heterogeneus platforms are the main keyworks in the last years. 


%This thesis follows such intent proposing a passive bilateral teleoperation architecture with kinesthetic feedback and virtual guiding mode to assist the operator performing the insertion phase.

%The architecture relays upon the Two-Layer teleoperation algorithm enhanced with energy evaluation in the task space and the possibility to scale energy between master and slave. It has been tested on two real manipulators available at the Altair robotics laboratory, University of Verona. 


\clearpage
\thispagestyle{empty}