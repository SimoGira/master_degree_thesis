\chapter{Abstract}
\markboth{Abstract}{Abstract}
Percutaneous therapies are constrained procedures in which the key issues are target access and visibility, tool manoeuvrability and physiological changes.
Unfortunately, human errors, imaging limitations, target uncertainty, tissue deformation and needle deflection are well known problems that contribute to needle misplacement in these procedures.
Real-time imaging techniques such as US and MRI can improve target visibility while robotic assistance has been proved to be a powerful tool to increase accuracy in needle positioning.
Nevertheless most clinical solutions require manual intervention for the insertion phase.
A fully automatic, robotic assisted system is desirable to reduce errors and increase accuracy in clinical procedures, such as biopsy, brachytherapy, anaesthesia and even stereotactic brain surgery; both development and validation of such a complex and integrated system are really challenging tasks.\\
Teleoperation may be a valuable alternative approach: by including the expertise of a clinician in the loop, we can demand the surgeon to address those main issues while preserving advantages from robotic assistance. Currently, robotic surgery addresses teleoperation mainly with unilateral schemas. The latter is due to the fact that the feedback introduced between the slave robot and the master console could introduce instability in the system.
This thesis follows such intent proposing a passive bilateral teleoperation architecture with kinesthetic feedback and virtual guiding mode to assist the operator performing the insertion phase.
The architecture relays upon the Two-Layer teleoperation algorithm enhanced with energy evaluation in the task space and the possibility to scale energy between master and slave. It has been tested on two real manipulators available at the Altair robotics laboratory, University of Verona. 


\clearpage
\thispagestyle{empty}